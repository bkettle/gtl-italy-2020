\documentclass[a4paper]{article}
\usepackage[utf8]{inputenc}
\usepackage[margin=1.5in]{geometry}
\pagenumbering{gobble}

\title{Resistivity Lab Handout}
\author{Ben Kettle}
\date{17 January 2020}

\begin{document}

\maketitle

\section{Introduction}
At the front of the class, we have a few lengths of wire, some thick and some thin. Your task is to find the resistivity of both a thick piece of wire and a thin piece of wire, and use these resistivities to decide if it's possible that they are the same material. In order to do this, you will need to make use of \textbf{Ohm's Law} and the \textbf{Definition of Resistivity}:

\begin{center}
    \begin{tabular}{cc}
    Ohm's Law & Definition of Resistivity \\
    $V = IR$ & $\rho = R\cdot \frac{A}{\ell}$ \\
    \end{tabular}
\end{center}

\section{Thin Wire}
First, go to the front of the room and get a piece of thin wire. Attach it between the two posts of one of the boards in the front, and attach the voltmeter and ammeter as discussed in the slides. Use the voltage and amperage readouts to calculate the resistance, and then measure the length. What is the resistivity of the material in the thin wire?
\vspace{25mm}

\section{Thick Wire}
Now, go to the front again and get a thick piece of wire and follow the same steps. What is the resistivity of the material in this wire?
\vspace{25mm}

\section{Conclusions}
How different are your values? Do you think these could be the same material?


\end{document}
